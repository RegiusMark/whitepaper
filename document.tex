\documentclass[12pt,a4paper]{article}
\usepackage[utf8]{inputenc}
\usepackage{array}
\usepackage{booktabs}
\usepackage{graphicx}
\usepackage[usenames,dvipsnames]{color}
\usepackage{dirtytalk}
\usepackage[margin=0.5in]{geometry}

\usepackage[maxnames=5]{biblatex}
\addbibresource{refs.bib}

\usepackage{hyperref}
\hypersetup{
  colorlinks,
  citecolor=blue,
  filecolor=black,
  linkcolor=[rgb]{0.1,0.3,0.7},
  urlcolor=black
}

\setlength{\parindent}{0pt}
\frenchspacing

\addtolength{\oddsidemargin}{0.2in}
\addtolength{\evensidemargin}{0in}
\addtolength{\textwidth}{-0.5in}
\addtolength{\topmargin}{-0.05in}

\begin{document}
  \begin{center}
    \includegraphics[width=90mm]{logo.png}\\
    \vspace{3mm}

    \vspace{3mm}
    \Huge{\textbf{REGIUS MARK}}\\
    \vspace{3mm}
    \normalsize{www.regiusmark.io}\\
    \normalsize{contact@regiusmark.io}\\
    \normalsize{\today}

    \vspace{30mm}
    \Large{\textbf{White Paper}}\\
    \Large{Ticker symbol: MARK}\\
    \vspace{30mm}

    \large{\textbf{Our mission is to bring utility to precious metals in}}\\
    \large{\textbf{an innovative blockchain solution --- Enter the Golden Age.}}

    \vspace*{\fill}
    \normalsize{© 2019 Regius Mark. All Rights Reserved.}
  \end{center}

  \newpage
  \begin{center}
    \textbf{DISCLAIMER OF LIABILITY}
  \end{center}
  The purpose of this White Paper is to present Regius Mark to potential coin
  holders and network operators. The information provided in this document may
  not be exhaustive, and it will not imply any elements of a contractual
  relationship. Its sole purpose is to provide relevant and reasonable
  information to potential coin holders who wish to mint or purchase Regius Mark
  while it is available on the open market.\\

  Nothing in this White Paper shall be deemed to constitute a prospectus of any
  sort or a solicitation for investment, nor does it in any way pertain to an
  offering or a solicitation of an offer to buy any securities in any
  jurisdiction. This document is not composed in accordance with, and is not
  subject to, laws or regulations of any jurisdiction, which are designed to
  protect investors.\\

  Regius Mark is not a security, and has not been registered under the
  Securities Act, the securities laws of any state of the United States or the
  securities laws of any country, including the securities laws of any
  jurisdiction in which a potential coin holder is a resident.\\

  Regius Mark is not intended for sale or use in any jurisdiction where sale or
  use of the coin may be prohibited.\\

  Regius Mark confers no other rights in any form, including but not limited to
  any ownership, distribution (including but not limited to profit), redemption,
  liquidation, proprietary (including all forms of intellectual property), or
  other financial or legal rights, other than those specifically described in
  the White Paper.

  \medskip
  \textit{Note:} This document is a work in progress and will be continuously
  updated as necessary.

  \newpage
  \tableofcontents
  \newpage

  \section{Introduction}
  Cryptocurrencies such as Bitcoin have been successful examples in creating a
  distributed network that users could trust, and many of these currencies have
  experienced a tremendous increase in value, regardless of having no backing.
  It has marked a significant beginning to the digital currency world that users
  can trust, becoming what many have called ``digital gold''.\\

  Currency markets are volatile where our token will be backed by physical gold
  providing stability and minimizing risk. The blockchain provides a means to
  secure funds digitally without weighing you down. Our technology is designed
  to be eco-friendly and easy to use for the average consumer.\\

  Decentralized networks have good intentions but we see it leads to issues such
  as scalability, performance, and inability to upgrade the software easily. Our
  centralized network allows us to scale, optimize for performance, and easily
  upgrade the system.\\

  The Regius Mark blockchain will contain one virtual asset with the name of
  MARK (gold). These virtual assets will be backed by physical gold,
  ensuring a stable market, rather than a volatile market.

  \section{Current Issues}
  There are a variety of issues in the financial system and existing
  cryptocurrencies that prevent wide-scale adoption by the user and
  merchants. There are risks involved for the merchant and the consumer.\\

  The common risk is a consequence of the volatile nature of the cryptocurrency
  markets. This volatility is notably caused by tokens that have no intrinsic
  value, relying on the trust and faith of its users to bring value. The
  volatility helped drive new technologies that are increasingly sophisticated
  for the merchants and users alike. However, both have to account for the
  ever-changing landscape of the token values. By physically stabilizing our
  token to gold, we get the benefits of the gold standard.\\

  When blockchains are under heavy load, the fees become expensive. The
  underlying technology is not designed to scale. Modern advances and research
  into algorithms allow us to create new technologies designed to be scalable
  for the world.\\

  The technology is still too difficult to use for the consumer and business
  owner alike. There's a problem when merchants have to use third-party payment
  processors to accept payments on the blockchain and users have to decide which
  wallet to use and learn how it works. We aim to have an intuitive wallet with
  native multi-signature support for consumers and SDKs designed to interact
  with the blockchain made available for merchants. The current lack of DX and
  UX is a hindrance to global adoption.\\

  Consumers want the cheapest fees when making transactions. Tech-savvy crypto
  users will hold a wide portfolio of tokens and decide which blockchain is the
  most cost effective to create a transaction. A novice will be forced to
  struggle with the expensive fees and time-to-time even the tech-savvy will
  have to as well.\\

  \newpage
  The Proof-of-Work algorithm is a dinosaur in the modern era of computer
  science. A Bitcoin specialist has determined it takes over an estimated 2.5
  gigawatts of electricity to mine. We are phasing out these expensive
  algorithms and replacing them with an eco-friendly system that is designed to
  scale and confirm transactions quickly. The efficiency is essential to enable
  a positive experience for the merchant and consumer.\\

  \section{Technology}
  We have seen blockchains provide a secure tamper-resistant database.
  Operations performed are deterministic allowing for the history to be
  verified. Any client will be able to synchronize the blockchain to validate
  past transactions as well as confirm the validity of future transactions.\\

  \subsection{Centralized}
  Being backed by physical assets requires a strong authority to strive
  properly. These assets may be handled by third-party brokers which present a
  proof of ownership to the network authorities. The authority will have
  permission to mint new tokens and store the associated documents including
  the distribution of these tokens.

  \subsection{Smart Contract VM}
  The virtual machine is used to execute smart contract bytecode. In Regius
  Mark, a stack machine is used with a forth-like scripting language. The
  bytecode supports a minimal set of opcodes required for financial services
  without adding unnecessary functionality to accomplish our goal. The opcodes
  used must be deterministic based on the current state of the blockchain.

  \subsection{Wallet}
  Wallets have been increasingly becoming easier to use by the user for
  basic transactions. However, these interfaces are too simple for the power
  user. Our interface will be versatile to the varying scenarios from simple
  tasks to the complex. Performing complex tasks by the power user should be as
  straight forward as the novice sending a transaction.\\

  The wallet experience and interface must be simple and intuitive for all
  users. Multi-signature wallets will be supported out of the box. A friend
  system can be used to create new wallets and sign a multi-signature
  transaction with the click of a few buttons. This system can also be used by
  merchants to accept payments and allow users to track who owns the address
  they sent the funds.\\

  For our expert users, advanced options and script builders will be available.
  The UI will be simple yet feature rich to avoid hindering expert users and
  ease new users getting into advanced smart contracts.

  \subsection{Security}
  Blockchains inherently need to be a fortress. All transactions are signed to
  prove the authenticity of the owner to perform an action on the blockchain.
  Regius Mark will support multi-signature wallets and smart contracts where
  higher levels of security are necessary.\\

  \newpage
  The centralized nature does not diminish the security of our infrastructure.
  The blockchain can be synchronized across the world in real time providing
  durability and tamper resistance as blockchain history cannot be rewritten.
  The master node will be using a multi-signature cold storage wallet and
  separate keys only for block production.

  \subsection{Transparency}
  Blockchains are naturally sequential and contain all the necessary data that
  pertains to the system. This allows us to easily distribute the block log
  without worrying about additional metadata. Any node operators will be able to
  synchronize the log and be able to remain in sync as new blocks arrive.

  \subsection{Minting}
  Blocks will be produced through a process called minting every three seconds.
  This process can only occur by use of a master node. The Regius Mark team will
  be the only master node on the network.\\

  Unlike traditional cryptocurrencies with block rewards, Regius Mark does not
  partake in the creation of tokens unless explicitly created by a master node
  during minting a new block. This is a necessary step to ensure that we never
  overcommit circulating tokens, in this way we never exceed our physical asset
  reserves.\\

  Minting transactions will contain data pertaining to a NI 43--101 report or
  any other document providing proof of ownership of physical gold.

  \subsection{Fees}
  Transaction fee costs start with a minimum fee. For every additional
  transaction accepted within the block window, the minimum fee is exponentially
  multiplied by the number of transactions accepted from the applicable address.
  The fee costs will reset back to the minimum fee after the block window is
  reset. The block window is reset when transactions are halted on the address
  for a period of time.\\

  In addition to the address fee mentioned in the above paragraph, there is an
  associated \say{global} network fee. The network fee works in the same way as
  the address based fee and protects the network from flooding via multiple
  addresses. The global fee is dynamically adjusted based on network usage.\\

  This quickly gets expensive for an attacker attempting to Denial-of-Service
  (DoS) the network, but allows flexibility for a normal user when waiting for
  the block window to reset back to the minimum fee is not an option.\\

  Our fees will remain low because of our dynamic fee model allowing us to
  adjust costs based on network usage. Any fees collected will be rewarded to
  the network operators.\\

  \newpage
  Sample based on transferring funds with a minimum fee of 0.0050 coins with a
  1.5000 multiplier:

  \vspace{3mm}
  \begin{tabular}{@{}lr@{}}
    Fees & Block Height     \\ \toprule
    0.0050 & 10             \\
    0.0075 & 11             \\
    0.0112 & 12             \\
    0.0050 & $\leftarrow{}$ block window reset $\rightarrow{}$ 20 \\ \midrule{}
    Total Fees & 0.0287     \\
    \bottomrule
  \end{tabular}

  \subsection{Message Signing}
  A signature is used to confirm the authenticity of the owner or owners of a
  particular message or document. Blocks produced by the minter and transactions
  created by the user are signed using the \textit{Ed25519}\cite{ed25519}
  algorithm.\\

  Ed25519 is a modern signing algorithm, it provides a similar protection level
  to NIST P-256 and has fast verification times with small signatures. This
  particular algorithm is capable of even faster batch verification using the
  Pippenger's method or the Bos-Coster method for scalar multiplication as
  mentioned in the Ed25519 paper.\\

  Ed25519 has fewer attack vectors, such as resistance to side-channel attacks
  and attacks from poor random number generator implementations. While
  non-deterministic algorithms can suffer from hardware fault attacks, it is
  extremely difficult to successfully execute. Even with a server running
  without ECC memory the attack isn't practical in any way over the
  internet.

  \subsection{Scaling}
  Scalability can be achieved through vertical or horizontal machine deployment.
  Vertically scaling the hardware is easy to maintain, but the costs may become
  infeasible as hardware requirements increase to process the influx of
  transactions being added to the network. We will use horizontal deployment
  through different types of nodes that serve a specific function to spread the
  workload across multiple machines.\\

  The majority of processing power required will be for transaction validation.
  We will use validator nodes to validate transactions being broadcasted to the
  network. The master node relies on the correctness of the validator to ensure
  that bad transactions cannot be added to the blockchain state.\\

  \subsubsection{Security}
  Each validator will contain an Ed25519 key utilized as an identity. The master
  node will send a challenge that the validator must sign to prove the machine
  is trusted. The link between the nodes must use the latest version of TLS to
  prevent any man in the middle or replay attacks.

  \newpage
  \section*{External Links}
  \begin{itemize}
    \item{\url{https://regiusmark.io}}
    \item{\url{https://github.com/RegiusMark}}
  \end{itemize}
  \printbibliography{}
\end{document}
